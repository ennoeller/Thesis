
% Thesis Abstract -----------------------------------------------------


%\begin{abstractslong}    %uncommenting this line, gives a different abstract heading
\begin{abstracts}        %this creates the heading for the abstract page
Smartphones are very common nowadays and people all around the world are using them in their everyday life. Even though mobile phones were originally invented as calling devices, smartphones allow the user to communicate in different ways including social media, which as of January 2014, 74\% of online adults use. In case they own a smartphone, they probably use social media on it as well, but with restrictions that come with the size of the device, affecting how we view content and also type. Typing on smartphones can be frustrating, but more so when the keyboard size prevents us from succeeding with authentication and we have to type the same text numerous times. This paper proposes a solution to such occurrences by using pattern recognition rather than typing. Patterns allow the screen to be used more efficiently, giving the user more room for accuracy errors. Survey results indicate that approaching authentication in this way is feasible.

%Both cloud computing and mobile computing domains have advanced rapidly and are the promising technologies for the near future. ...



\end{abstracts}
%\end{abstractlongs}


% ---------------------------------------------------------------------- 
