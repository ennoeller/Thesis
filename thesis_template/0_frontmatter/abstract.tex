\noindent\textbf{\large Simplifying Mobile Social Media Authentication On Android}
\vspace*{2ex}
{\flushleft{\textbf{Abstract:}}}

\noindent Nowadays, smartphones are very common and people all around the world are using them in their everyday life. Even though mobile phones were originally invented as calling devices, smartphones allow the user to communicate in different ways including social media, instant messaging, recording and watching videos, etc. Recent statistics as of January 2014 claim, that 74\% of online adults use social media. In case they own a smartphone, they probably use social media on it as well, but with restrictions that come with the size of the device, affecting how we view content and also type. Typing on smartphones can be frustrating, but more so when the keyboard size prevents us from succeeding with authentication and we have to type the same text numerous times, which can lead to shorter passwords decreasing the security of the accounts. This paper proposes a solution to such occurrences by using pattern recognition rather than typing. Patterns allow the screen to be used more efficiently, giving the user more room for accuracy errors. Survey results indicate that approaching authentication in this way is feasible.

%Recent statistics presented by <who made it>

\vspace*{2ex}

{\flushleft{\textbf{Keywords:} Mobile, Android, Authentication, Pattern}}

\vspace*{3ex}

\noindent\textbf{\large Mobiilse Sotsiaalmeedia Autentimise Lihtsustamine Androidil}
\vspace*{2ex}
{\flushleft{\textbf{L\"{u}hikokkuv\~{o}te:}}}

\noindent T\"{a}nap\"{a}eval on nutitelefonid v\"{a}ga levinud ja k\~{o}ikjal maailmas inimesed kasutavad neid oma igap\"{a}eva elus. Kuigi mobiiltelefonid loodi algselt helistamiseks, nutitelefonid v\~{o}imaldavad kasutajal suhelda erinevatel viisidel, nende seas sotsiaalmeedia, kiirs\~{o}numid, lindistada ja vaadata videoid jne. Hiljutised uuringud jaanuarist 2014 v\"{a}idavad, et 74\% internetti kasutavatest t\"{a}isealistest tarvitavad ka sotsiaalset meediat. Juhul, kui need isikud omavad nutitelefoni, on t\~{o}en\"{a}osus, et nad kasutavad sotsiaalmeediaid ka oma nutiseadmel, kuid piirangutega, mis tulenevad seadme suurusest. Suurus m\~{o}jutab, kuidas me infot vaatame ja teksti sisestame. Tr\"{u}kkimine nutitelefonil v\~{o}ib osutuda masendavaks, seda enam, kui klaviatuuri suurus takistab meil autentimise edu ja me peame sama teksti sisestama mitmeid kordi. Sellised olukorrad v\~{o}ivad viia l\"{u}hemate paroolide kasutamiseni, mis omakorda v\"{a}hendab meie kontode turvalisust. Antud t\"{o}\"{o} pakub v\"{a}lja lahenduse sellistele olukordadele kasutades tr\"{u}kkimise asemel mustreid. Mustrid v\~{o}imaldavad efektiivsemat ekraani kasutust, mis annavad kasutajale rohkem ruumi t\"{a}psuse vigade v\"{a}ltimiseks. Uuringu tulemused n\"{a}itavad, et selline l\"{a}henemine autentimisele on v\~{o}imalik.

\vspace*{2ex}

{\flushleft{\textbf{V\~{o}tmes\~{o}nad:} Mobiilne, Android, Autentimine, Muster}}
