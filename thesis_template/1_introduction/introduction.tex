
% this file is called up by thesis.tex
% content in this file will be fed into the main document

%: ----------------------- introduction file header -----------------------
\chapter{Introduction}

% the code below specifies where the figures are stored
\ifpdf
    \graphicspath{{1_introduction/figures/PNG/}{1_introduction/figures/PDF/}{1_introduction/figures/}}
\else
    \graphicspath{{1_introduction/figures/EPS/}{1_introduction/figures/}}
\fi

% ----------------------------------------------------------------------
%: ----------------------- introduction content -----------------------
% ----------------------------------------------------------------------



%: ----------------------- HELP: latex document organisation
% the commands below help you to subdivide and organise your thesis
%    \chapter{}       = level 1, top level
%    \section{}       = level 2
%    \subsection{}    = level 3
%    \subsubsection{} = level 4
% note that everything after the percentage sign is hidden from output

Social media has become a part or our life, whether we communicate by it, express ourselves or use it for entertainment. Originally these media sites were developed for use in PC, but over the years as smartphones are becoming more common and capable, most of these sites have moved closer to us, into our pockets. With the innovation of smartphone social media applications, one would guess that the way we authenticate ourselves would change, to accommodate these devices, but so far they are mistaken.


\subsection{Motivation}
Applications on Android are mostly configured to keep the user signed in or memorize the credentials, though there are still applications that have not gone or are still implementing these methods, like Skype. \footnote{http://www.skype.com/en/} \footnote{http://www.androidauthority.com/skype-update-v5-5-makes-it-easy-to-sign-in-625877/}. Implementing automatic authentication may increase the usability of an applications, it also raises a security concern, where the device holder has access to the data, which normally would be hidden behind authentication requirements.

Realising this security issue, one might keep his applications signed out, whenever not using the device or log out purposely when lending the device to somebody. After doing so, the user would have to authenticate all the applications the hard way - typing. Given the inconvenience of the conventional typing authentication process on our devices, one might use the application less.


\subsection{Contributions}





\subsection{Outline}
%for example
%\noindent \textbf{Chapter 2}: discusses the state of the art addressed by this thesis. 

%\noindent \textbf{Chapter 3}: explains the problems regarding the combination and the invocation of cloud services from the mobile phone. 

Brief introduction of each chapter %two of three lines


