% this file is called up by thesis.tex
% content in this file will be fed into the main document

%: ----------------------- name of chapter  -------------------------
\chapter{Problem Statement} % top level followed by section, subsection

In the few years previous to 2010, tablets started to circle the market and nobody saw exactly how it would affect the devices we own. In 2010 Steve Jobs, the co-founder of Apple Incorporation, predicted that tablets would overtake PCs (personal computer). Slowly tablets have reduced the sales gap and in 2015 that prediction will probably come true\footnote[11]{http://www.extremetech.com/computing/185937-in-2015-tablet-sales-will-finally-surpass-pcs-fulfilling-steve-jobs-post-pc-prophecy}.

Though smartphones are becoming more common as means to use social media, the authentication process remains the same as in personal computers. That in mind, when we look at these devices, we can immediately notice the size and lack of peripherals compared to PCs, raising the concern of input difficulty. In particular, inputting complicated passwords, that require precision to the last letter. 

Applications in Android are mostly configured to keep the user signed in, but there are those who like to keep their devices logged out of apps for security reasons or for the fact that they share a device. In such case the authentication process is inevitable.



%: ----------------------- paths to graphics ------------------------

% change according to folder and file names
\ifpdf
    \graphicspath{{X/figures/PNG/}{X/figures/PDF/}{X/figures/}}
\else
    \graphicspath{{X/figures/EPS/}{X/figures/}}
\fi

\section{Research Question}
The proposed solution to the problem of conventional username-password authentication method is to modify traditional input mechanism to better fit the characteristics of a smartphone. These devices have other mechanisms to capture data from the user, to simplify the input data process for recurrent authentication process. Mechanism, the screen of the device, can be used to introduce gestures from the user as patterns. 

This thesis aims to evaluate whether a pattern based method of authentication is more user-friendly than the traditional approach. To do so, such a mechanism is developed and implemented for a mobile application as proof of concept. A survey is then conducted to compare this method to the traditional approach in terms of simplicity and repulsiveness. 

\section{Summary}

As mobile devices are slowly taking over from PCs, applications need to consider the characteristics of a smartphone and evolve accordingly. This thesis attempts to find a solution to the level of difficulty that traditional authentication methods have on mobile devices. A pattern based authentication method is developed to raise the simplicity of the process and is implemented in an app. A survey is conducted to determine whether the proposed method is more suitable for mobile devices. The next chapter describes the architecture of the created solution. 



% ---------------------------------------------------------------------------
%: ----------------------- end of thesis sub-document ------------------------
% ---------------------------------------------------------------------------

